\documentclass[a4paper]{article}

\usepackage{amsmath, amssymb}

\pdfsuppresswarningpagegroup=1

\begin{document}

\section{Vector: 3 perspectives - Physics, Math, CS }

\begin{align*}
	\begin{bmatrix}
		7 & 8 & 6 \\
		6 & 8 & 4 \\
		7 & 5 & 4 \\
	\end{bmatrix}
\end{align*}

\subsection{physics}
\begin{itemize}
	\item vectors are arrows in space with length and direction
	\item so vectors can be moved around in space without any issues
\end{itemize}

\subsection{cs}
\begin{itemize}
	\item vectors are ordered lists of numbers
\end{itemize}

\subsection{math}
\begin{itemize}
	\item seeks to generalize both of these views and defines vector operations such as
	      addition and multiplication
\end{itemize}

\subsection{In linear algebra}
\begin{itemize}
	\item graham suggests to view vector as arrow with tail always fixed at origin
	\item maybe i can imagine a vector as a operation of shifting the origin
	\item thus when we do vector addition we will start the second vector from the
	      start of first vector, since the origin has been shifted by the first vector
	\item Note: "but" it could really be shifting of all the points in the coordinate system
	      (as told by graham)
\end{itemize}

\subsection{Scalars}
\begin{itemize}
	\item The numbers that we multiply the vector with to scale the vector
	      in its original direction
	\item since it's used frequently it just interchangeable with number
\end{itemize}


\newpage
\section{Linear Combinations, Spans, Basis Vectors}

\subsection{Vector Coordinates}
\begin{itemize}
	\item vector coordinates are the number numbers present in a vector
	\item Each of the vector coordinates is also a scalar
	      that scales the unit vectors of the coordinate system $\hat{i}, \hat{j}$
\end{itemize}

\subsection{Span}
Span of $\vec{v} \text{ and } \vec{w}$ are the set of all of their linear combinations
\[
	a \vec{v} + b \vec{w}
\]

Note: Its common to think a collection of vectors as points, due to clustering/noise

\subsection{Linearly dependent Vectors}
\begin{itemize}
	\item If the third vector can be formed by linear combination of the other vectors
	      then the vectors are said to be linearly dependent
	\item If we cannot get a vector by linear combination of other vectors then
	      then those vectors are called as linearly independent
\end{itemize}

\subsection{Basis}
\begin{itemize}
	\item basis of a vector space is a set of linearly independent vectors
	      that span the full space
\end{itemize}

\end{document}
